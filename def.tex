% Codificação UTF-8 para Unicode, se não estivermos na engine LuaLaTeX
\ifx\directlua\undefined
	\usepackage[utf8]{inputenc}
\fi

\usepackage{graphicx} % Inclusão de gráficos
\usepackage{hyperref} % Hiperlinks e referências

\usepackage[sc]{mathpazo} % Fonte Palatino com pequenas maiúsculas
\usepackage[T1]{fontenc} % Codificação T1 para fontes
\usepackage{microtype} % Microtipografia
\linespread{1.05} % Ajuste de espaçamento entre linhas

\usepackage{siunitx} % Unidades SI
	\sisetup{output-decimal-marker = {,}} % Vírgula como separador decimal
	\sisetup{range-phrase = \text{--}} % En-dash para intervalos

\usepackage[brazilian]{babel} % Idioma português brasileiro

% Geometria da página
\usepackage[
	hmarginratio=1:1,
	top=32mm,
	columnsep=21pt
]{geometry}

% Legendas personalizadas
\usepackage[
	hang,
	small,
	labelfont=bf,up,
	textfont=it,up
]{caption}

\usepackage{booktabs} % Formatação de tabelas
\usepackage{lettrine} % Letras capitulares

\usepackage{enumitem} % Listas personalizadas
	\setlist[itemize]{noitemsep} % Remove espaçamento em listas

\usepackage{abstract} % Ambiente de resumo
	\renewcommand{\abstractnamefont}{\normalfont\bfseries} % "Resumo" em negrito
	\renewcommand{\abstracttextfont}{\normalfont\small\itshape} % Texto em itálico pequeno

\usepackage{titlesec} % Títulos de seções
	\renewcommand\thesection{\Roman{section}} % Numeração romana maiúscula
	\renewcommand\thesubsection{\roman{subsection}} % Numeração romana minúscula
	% Formato centrado para seções
	\titleformat{\section}[block]{\large\scshape\centering}{\thesection.}{1em}{}
	% Formato para subseções
	\titleformat{\subsection}[block]{\large}{\thesubsection.}{1em}{}

\usepackage{fancyhdr} % Cabeçalhos e rodapés
	\pagestyle{fancy} % Estilo com cabeçalhos
	\fancyhead{} % Limpa cabeçalhos
	\fancyfoot{} % Limpa rodapés
	\fancyhead[C]{Experimento 4: Atrito Cinético} % Texto no cabeçalho
	\fancyfoot[C]{\thepage} % Numeração no rodapé

\usepackage{titling} % Título do documento

\usepackage{amsmath, amsthm, amssymb, amsfonts} % Matemática AMS

\usepackage{minted} % Código com realce
\usepackage[dvipsnames]{xcolor} % Definição de cores

% Configurações para Unicode em matemática e para os ambientes do `minted`
\usepackage{unicode-math} % Suporte Unicode para matemática
\usepackage{newunicodechar} % Caracteres Unicode personalizados
\ifx\directlua\undefined
\else
	\defaultfontfeatures{Scale = MatchLowercase} % Escala de fontes
	\setmainfont{CMU Serif}[Scale = 1.0] % Fonte serifada
	\setsansfont{CMU Sans Serif} % Fonte sans-serif
	\setmonofont{CMU Typewriter Text} % Fonte monoespaçada
	\setmathfont{Latin Modern Math} % Fonte matemática
\fi

\usepackage{pbalance} % Balanceia a disposição das colunas na última página

\pretitle{\begin{center}\Huge\bfseries} % Formatação do título
\posttitle{\end{center}} % Fecha formatação
