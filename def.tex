\ifx\directlua\undefined
	\usepackage[utf8]{inputenc} % Unicode
\fi

\usepackage{blindtext} % Pra testar a formatação
\usepackage{graphicx} % Imagens

\usepackage{hyperref} % Pra ajudar a lidar com referências e hyperlinks

\usepackage[sc]{mathpazo} % Usando uma fonte diferente para o documento
	\usepackage[T1]{fontenc} % Use 8-bit encoding that has 256 glyphs
	\linespread{1.05} % Aumentando o espaçamento entre as linhas (a fonte não fica tão legal com o espaçamento padrão)
	\usepackage{microtype} % Não lembro o que isso faz

\usepackage{siunitx} % Facilita o uso das unidades do SI
	\sisetup{output-decimal-marker = {,}} % Configura a vírgula como o separador decimal
	\sisetup{range-phrase = \text{--}}

\usepackage[brazilian]{babel} % Regras tipográficas

\usepackage[hmarginratio=1:1,top=32mm,columnsep=21pt]{geometry} % Margens do documento
\usepackage[hang, small,labelfont=bf,up,textfont=it,up]{caption} % Legendas customizadas pra tabelas e imagens
\usepackage{booktabs} % Não lembro

\usepackage{lettrine} % Pra aumentar o tamanho da primeira letra do texto (o que não é remotamente necessário pra nada)

\usepackage{enumitem} % Listas customizadas
	\setlist[itemize]{noitemsep} % Pra tornar as listas mais compactas

\usepackage{abstract}
	\renewcommand{\abstractnamefont}{\normalfont\bfseries} % Deixa o "Resumo" em negrito
	\renewcommand{\abstracttextfont}{\normalfont\small\itshape} % Deixa o conteúdo do resumo em itálico

\usepackage{titlesec} % Customização do título
	\renewcommand\thesection{\Roman{section}} % Números romanos para as secções
	\renewcommand\thesubsection{\roman{subsection}} % Para as subsecções também
	\titleformat{\section}[block]{\large\scshape\centering}{\thesection.}{1em}{} % Muda a aparência do título das secções
	\titleformat{\subsection}[block]{\large}{\thesubsection.}{1em}{} % Muda a aparência do título das subsecções

\usepackage{fancyhdr} % Cabeçalho
	\pagestyle{fancy} % Cabeçalho em todas as páginas
	\fancyhead{}
	\fancyfoot{}
	\fancyhead[C]{Experimento 4: Atrito Cinético}%Lucca Pellegrini $\bullet$ \today} % Custom header text
	\fancyfoot[C]{\thepage} % Custom footer text

\usepackage{titling} % Customização do título

\usepackage{amsmath, amsthm, amssymb, amsfonts} % AMS-TeX pra equações (em geral) mais bonitas e pra umas outras coisas

\usepackage{minted} % Para incluir código fonte
\usepackage[dvipsnames]{xcolor} % Para a definição de cores

% Configurações para caracteres desconhecidos pelo `minted`
\usepackage{unicode-math}
\usepackage{newunicodechar}
	\defaultfontfeatures{Scale = MatchLowercase}
	\setmainfont{CMU Serif}[Scale = 1.0]
	\setsansfont{CMU Sans Serif}
	\setmonofont{CMU Typewriter Text}
	\setmathfont{Latin Modern Math}

\usepackage{csquotes} % Acho que isso serve pra facilitar lidar com citações, mas não tenho certeza

\pretitle{\begin{center}\Huge\bfseries} % Formatação do título
\posttitle{\end{center}} % Fecha a formatação
