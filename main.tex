\documentclass[10pt,oneside,twocolumn,a4paper]{article} % Duas colunas pra caber mais texto

% Lab 4: Atrito Cinético
% Copyright 2025 Lucca Pellegrini <lucca@verticordia.com>
%           2025 Pedro Caiafa Andrade
%           2025 Cauã Diniz Armani
%           2025 Amanda Canizela
%           2025 Júlia de Mello Teixeira
%
% Permission to use, copy, modify, and/or distribute this software for any
% purpose with or without fee is hereby granted, provided that the above
% copyright notice and this permission notice appear in all copies.
%
% THE SOFTWARE IS PROVIDED “AS IS” AND THE AUTHOR DISCLAIMS ALL WARRANTIES WITH
% REGARD TO THIS SOFTWARE INCLUDING ALL IMPLIED WARRANTIES OF MERCHANTABILITY
% AND FITNESS. IN NO EVENT SHALL THE AUTHOR BE LIABLE FOR ANY SPECIAL, DIRECT,
% INDIRECT, OR CONSEQUENTIAL DAMAGES OR ANY DAMAGES WHATSOEVER RESULTING FROM
% LOSS OF USE, DATA OR PROFITS, WHETHER IN AN ACTION OF CONTRACT, NEGLIGENCE OR
% OTHER TORTIOUS ACTION, ARISING OUT OF OR IN CONNECTION WITH THE USE OR
% PERFORMANCE OF THIS SOFTWARE.

\ifx\directlua\undefined
\usepackage[utf8]{inputenc} % Unicode
\fi
\usepackage{blindtext} % Pra testar a formatação
\usepackage{graphicx} % Imagens

\usepackage{hyperref} % Pra ajudar a lidar com referências e hyperlinks

\usepackage[sc]{mathpazo} % Usando uma fonte diferente para o documento
        \usepackage[T1]{fontenc} % Use 8-bit encoding that has 256 glyphs
        \linespread{1.05} % Aumentando o espaçamento entre as linhas (a fonte não fica tão legal com o espaçamento padrão)
        \usepackage{microtype} % Não lembro o que isso faz

\usepackage{siunitx} % Facilita o uso das unidades do SI
        \sisetup{output-decimal-marker = {,}} % Configura a vírgula como o separador decimal
        \sisetup{range-phrase = \text{--}}

\usepackage[brazilian]{babel} % Regras tipográficas

\usepackage[hmarginratio=1:1,top=32mm,columnsep=21pt]{geometry} % Margens do documento
\usepackage[hang, small,labelfont=bf,up,textfont=it,up]{caption} % Legendas customizadas pra tabelas e imagens
\usepackage{booktabs} % Não lembro

\usepackage{lettrine} % Pra aumentar o tamanho da primeira letra do texto (o que não é remotamente necessário pra nada)

\usepackage{enumitem} % Listas customizadas
        \setlist[itemize]{noitemsep} % Pra tornar as listas mais compactas

\usepackage{abstract}
        \renewcommand{\abstractnamefont}{\normalfont\bfseries} % Deixa o "Resumo" em negrito
        \renewcommand{\abstracttextfont}{\normalfont\small\itshape} % Deixa o conteúdo do resumo em itálico

\usepackage{titlesec} % Customização do título
        \renewcommand\thesection{\Roman{section}} % Números romanos para as secções
        \renewcommand\thesubsection{\roman{subsection}} % Para as subsecções também
        \titleformat{\section}[block]{\large\scshape\centering}{\thesection.}{1em}{} % Muda a aparência do título das secções
        \titleformat{\subsection}[block]{\large}{\thesubsection.}{1em}{} % Muda a aparência do título das subsecções

\usepackage{fancyhdr} % Cabeçalho
        \pagestyle{fancy} % Cabeçalho em todas as páginas
        \fancyhead{}
        \fancyfoot{}
        \fancyhead[C]{Experimento 4: Atrito Cinético}%Lucca Pellegrini $\bullet$ \today} % Custom header text
        \fancyfoot[C]{\thepage} % Custom footer text

\usepackage{titling} % Customização do título

\usepackage{url} % Pra ajudar a lidar com urls chatos

\usepackage{amsmath, amsthm, amssymb, amsfonts} % AMS-TeX pra equações (em geral) mais bonitas e pra umas outras coisas

\usepackage{minted} % Para incluir código fonte
\usepackage[dvipsnames]{xcolor} % Para a definição de cores

% Configurações para caracteres desconhecidos pelo `minted`
\usepackage{unicode-math}
\usepackage{newunicodechar}
\defaultfontfeatures{Scale = MatchLowercase}
\setmainfont{CMU Serif}[Scale = 1.0]
\setsansfont{CMU Sans Serif}
\setmonofont{CMU Typewriter Text}
\setmathfont{Latin Modern Math}

\usepackage{csquotes} % Acho que isso serve pra facilitar lidar com citações, mas não tenho certeza

\setlength{\droptitle}{-4\baselineskip} % Sobe o título um pouco pra economizar espaço

\pretitle{\begin{center}\Huge\bfseries} % Formatação do título
\posttitle{\end{center}} % Fecha a formatação

%------------------------------------------------------------------------------------
% TÍTULO
%------------------------------------------------------------------------------------

\title{Atrito Cinético}
\author{
        \textsc{Amanda Canizela} \\[1ex] % Aqui, pode-se colocar uma footnote com o comando \thanks{alguma coisa}
        \normalsize{847501}
        \and % Mais autores, se necessário
        \textsc{Cauã Diniz} \\[1ex]
        \normalsize{843988}
        \and
        \textsc{Júlia de Mello} \\[1ex]
        \normalsize{819881}
        \and
        \textsc{Lucca Pellegrini} \\[1ex]
        \normalsize{842986}
        \and
        \textsc{Pedro Vitor Caiafa} \\[1ex]
        \normalsize{842843}
}

\date{\today} % Data
\renewcommand{\maketitlehookd}{         % O resumo
        \begin{abstract}
        \noindent
				Este experimento didático visa determinar o coeficiente de
				atrito cinético entre um bloco de madeira e uma superfície
				inclinada, utilizando uma rampa ajustada a $\SI{30}{\degree}$.
				Através de medições de tempo e posição com sensores de luz,
				ajustamos os dados a uma equação quadrática para calcular a
				aceleração e, subsequentemente, o coeficiente de atrito. Os
				resultados indicam um valor de $\mu_k = 0,093 \pm 0,006$,
				compatível com materiais como madeira e plástico, destacando a
				eficácia do método e a importância das incertezas experimentais
				na análise física.
        \end{abstract}
}

%------------------------------------------------------------------------------------

\begin{document}
%\pagenumbering{gobble} % Esconde o número de página, porque o documento é pequeno


% Renderiza o título
\maketitle

%------------------------------------------------------------------------------------
% TEXTO
%------------------------------------------------------------------------------------

\section{Motivação}

O atrito é uma das forças mais presentes no cotidiano e desempenha papel
fundamental em diversos sistemas físicos e tecnológicos. Entender como
determinar o coeficiente de atrito cinético em um plano inclinado nos permite
modelar desde situações simples, como o movimento de objetos em superfícies
inclinadas, até aplicações em engenharia e transporte, onde a previsão do
atrito é essencial para a segurança e eficiência. O experimento realizado
possibilita, portanto, compreender de forma prática como medir esse coeficiente
e quais fatores influenciam sua determinação.

\section{Materiais}

Para a realização do experimento utilizamos:

\begin{itemize}
    \item Rampa com ângulo ajustável, fixada em $\SI{30 \pm 0.5}{\degree}$;
    \item Dois sensores de luz com cronômetro digital acoplado;
    \item Bloco de massa $\SI{22}{\gram}$;
	\item Régua com marcações de $\SIrange{0.05}{0.32}{\metre};$
	\item Um pequeno script em \textit{Python}\footnote{Veja o
		Anexo~\ref{sec:python}.} para análise de dados, geração do gráfico, e
		ajuste de curva.
\end{itemize}

\section{Métodos}

O bloco foi liberado a partir do repouso sobre a rampa inclinada de $30^\circ$.
Com o auxílio dos sensores de luz, medimos o tempo $t$ que o bloco levava para
percorrer intervalos de $\SI{0,05}{\metre}$. Os dados coletados foram ajustados
no script \textit{Python} a uma função linear do tipo:

\begin{align}
	x(t) &= \frac{at^2}{2} + v_0t, \nonumber \\
	x(t) &= At^2 + Bt
\end{align}

onde consideramos $t^2$ como nossa variável independente, $A = \frac{a}{2}$, e
a posição inicial como $x(0) = \SI{0}{\meter}$.

\section{Resultados}

A tabela~\ref{tbl:dados} mostra os valores medidos de $\Delta x$, $t$ e
$t^2$, sem arredondamento:

\begin{table}[ht!]
	\centering
		\begin{tabular}{|| c | c | c ||}
		\hline
		$\Delta x$ ($\SI{}{\metre}$) & $t$ ($\SI{}{\second}$) & $t^2$ ($\SI{}{\second^2}$) \\
		\hline\hline
		$0,05$ & $0,10225$ & $0,0104500625$ \\
		$0,10$ & $0,18040$ & $0,0325441600$ \\
		$0,15$ & $0,22880$ & $0,0522944400$ \\
		$0,20$ & $0,27200$ & $0,0739840000$ \\
		$0,25$ & $0,31830$ & $0,1012608900$ \\
		$0,30$ & $0,35640$ & $0,1270100000$ \\
		$0,35$ & $0,39620$ & $0,1569816400$ \\
		\hline
	\end{tabular}
	\caption{Valores de $\Delta x$, $t$ e $t^2$ medidos sem arredondamento.}
	\label{tbl:dados}
\end{table}

Do ajuste linear (veja a figura~\ref{fig:kinetic_friction}), obtivemos:

\begin{align}\label{eq:accel}
	A &= \left( 2,056 \pm 0,066 \right) \si{\metre\per\second\squared}, \nonumber \\
	\therefore 2A &= a \Rightarrow \nonumber \\
	a &= \left( 4,1 \pm 0,1 \right) \si{\metre\per\second\squared}.
\end{align}

Sabemos que, para um corpo em um plano inclinado com atrito:

\begin{align}\label{eq:mu}
	\frac{a}{g} &= \sin\theta - \mu_k \cos\theta, \nonumber \\
	\mu_k &= \frac{\sin\theta - \frac{a}{g}}{\cos\theta}.
\end{align}

\section{Incertezas}

Para calcular a incerteza, determinamos os valores máximo e mínimo possíveis em
(\ref{eq:mu}) da seguinte forma:

\begin{align*}
	\mu_{k_\text{max}} &= \frac{\sin(\theta + \Delta\theta) - \frac{a - \Delta a}{g + \Delta g}}{\cos(\theta + \Delta\theta)}, \\
	\mu_{k_\text{min}} &= \frac{\sin(\theta - \Delta\theta) - \frac{a + \Delta a}{g - \Delta g}}{\cos(\theta - \Delta\theta)}.
\end{align*}


Substituindo $g = \SI{9.8}{\meter\per\second\squared}$, $\Delta g =
\SI{0.05}{\meter\per\second\squared}$, $\theta = \SI{30}{\degree}$,
$\Delta\theta = \SI{0.5}{\degree}$, e os valores de $a$ e $\Delta a$ obtidos em
(\ref{eq:accel}), temos:

\begin{align*}
	\mu_{k_\text{max}} &\approx 0,11279, \\
	\mu_{k_\text{min}} &\approx 0,07296, \\
	\overline\mu_k = \frac{\mu_{k_\text{max}} + \mu_{k_\text{min}}}{2} &\approx 0,09287, \\
	\Delta\mu_k = \frac{\mu_{k_\text{max}} - \mu_{k_\text{min}}}{2\sqrt{3}} &\approx 0,00558.
\end{align*}

\section{Conclusão}

O valor do coeficiente de atrito cinético encontrado foi:

\begin{equation*}
	\mu_k = 0,093 \pm 0,006.
\end{equation*}

Esse valor é compatível com superfícies de madeira e plástico em contato, e
demonstra a importância de considerar as incertezas experimentais. O método
adotado, via ajuste linear da posição em função do tempo ao quadrado,
mostrou-se eficaz para determinar a aceleração e, consequentemente, o
coeficiente de atrito cinético no sistema analisado.

%------------------------------------------------------------------------------------


\begin{onecolumn}
\begin{figure}[h]
        \centering
        \includegraphics[width=15cm]{kinetic_friction_plot.pdf}
		\caption{
			Gráfico de posição $\Delta x$ em função do tempo ao quadrado $t^2$.
			O ajuste linear fornece $A = (2,056 \pm 0,066)$\,m/s$^2$ e $B =
			(0,037 \pm 0,006)$\,m/s, onde $A = \frac{a}{2}$ e $B = v_0$.
		}
        \label{fig:kinetic_friction}
\end{figure}
\end{onecolumn}

\newpage
%------------------------------------------------------------------------------------
% REFERÊNCIAS
%------------------------------------------------------------------------------------

%\bibliographystyle{unsrt}
%\bibliography{ref}

%------------------------------------------------------------------------------------

\appendix

\section{Código Fonte do Script Python}\label{sec:python}

Segue abaixo o script escrito para esse experimento. O módulo auxiliar
\texttt{PhysPlot}, de minha autoria, pode ser encontrado no meu
GitHub.\footnote{\url{https://github.com/lucca-pellegrini/PhysPlot/blob/v0.1.1}}

\inputminted[
	linenos = true,
	style = staroffice,
	fontsize = \footnotesize,
	frame = lines,
	framesep=2em,
	rulecolor=\color{Gray},
	label=\fbox{\color{Black}kinetic\_friction.py},
	labelposition=topline
]{python}{kinetic_friction.py}

\end{document}
