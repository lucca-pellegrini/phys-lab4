\documentclass[10pt,oneside,twocolumn,a4paper]{article}
% Codificação UTF-8 para Unicode, se não estivermos na engine LuaLaTeX
\ifx\directlua\undefined
	\usepackage[utf8]{inputenc}
\fi

\usepackage{graphicx} % Inclusão de gráficos
\usepackage{hyperref} % Hiperlinks e referências

\usepackage[sc]{mathpazo} % Fonte Palatino com pequenas maiúsculas
\usepackage[T1]{fontenc} % Codificação T1 para fontes
\usepackage{microtype} % Microtipografia
\linespread{1.05} % Ajuste de espaçamento entre linhas

\usepackage{siunitx} % Unidades SI
	\sisetup{output-decimal-marker = {,}} % Vírgula como separador decimal
	\sisetup{range-phrase = \text{--}} % En-dash para intervalos

\usepackage[brazilian]{babel} % Idioma português brasileiro

% Geometria da página
\usepackage[
	hmarginratio=1:1,
	top=32mm,
	columnsep=21pt
]{geometry}

% Legendas personalizadas
\usepackage[
	hang,
	small,
	labelfont=bf,up,
	textfont=it,up
]{caption}

\usepackage{booktabs} % Formatação de tabelas
\usepackage{lettrine} % Letras capitulares

\usepackage{enumitem} % Listas personalizadas
	\setlist[itemize]{noitemsep} % Remove espaçamento em listas

\usepackage{abstract} % Ambiente de resumo
	\renewcommand{\abstractnamefont}{\normalfont\bfseries} % "Resumo" em negrito
	\renewcommand{\abstracttextfont}{\normalfont\small\itshape} % Texto em itálico pequeno

\usepackage{titlesec} % Títulos de seções
	\renewcommand\thesection{\Roman{section}} % Numeração romana maiúscula
	\renewcommand\thesubsection{\roman{subsection}} % Numeração romana minúscula
	% Formato centrado para seções
	\titleformat{\section}[block]{\large\scshape\centering}{\thesection.}{1em}{}
	% Formato para subseções
	\titleformat{\subsection}[block]{\large}{\thesubsection.}{1em}{}

\usepackage{fancyhdr} % Cabeçalhos e rodapés
	\pagestyle{fancy} % Estilo com cabeçalhos
	\fancyhead{} % Limpa cabeçalhos
	\fancyfoot{} % Limpa rodapés
	\fancyhead[C]{Experimento 4: Atrito Cinético} % Texto no cabeçalho
	\fancyfoot[C]{\thepage} % Numeração no rodapé

\usepackage{titling} % Título do documento

\usepackage{amsmath, amsthm, amssymb, amsfonts} % Matemática AMS

\usepackage{minted} % Código com realce
\usepackage[dvipsnames]{xcolor} % Definição de cores

% Configurações para Unicode em matemática e para os ambientes do `minted`
\usepackage{unicode-math} % Suporte Unicode para matemática
\usepackage{newunicodechar} % Caracteres Unicode personalizados
	\defaultfontfeatures{Scale = MatchLowercase} % Escala de fontes
	\setmainfont{CMU Serif}[Scale = 1.0] % Fonte serifada
	\setsansfont{CMU Sans Serif} % Fonte sans-serif
	\setmonofont{CMU Typewriter Text} % Fonte monoespaçada
	\setmathfont{Latin Modern Math} % Fonte matemática

\usepackage{pbalance} % Balanceia a disposição das colunas na última página

\pretitle{\begin{center}\Huge\bfseries} % Formatação do título
\posttitle{\end{center}} % Fecha formatação


%==============================================================================
% Lab 4: Atrito Cinético
% Copyright 2025 Lucca Pellegrini <lucca@verticordia.com>
%           2025 Pedro Caiafa Andrade
%           2025 Cauã Diniz Armani
%           2025 Amanda Canizela
%           2025 Júlia de Mello Teixeira
%
% Permission to use, copy, modify, and/or distribute this software for any
% purpose with or without fee is hereby granted, provided that the above
% copyright notice and this permission notice appear in all copies.
%
% THE SOFTWARE IS PROVIDED “AS IS” AND THE AUTHOR DISCLAIMS ALL WARRANTIES WITH
% REGARD TO THIS SOFTWARE INCLUDING ALL IMPLIED WARRANTIES OF MERCHANTABILITY
% AND FITNESS. IN NO EVENT SHALL THE AUTHOR BE LIABLE FOR ANY SPECIAL, DIRECT,
% INDIRECT, OR CONSEQUENTIAL DAMAGES OR ANY DAMAGES WHATSOEVER RESULTING FROM
% LOSS OF USE, DATA OR PROFITS, WHETHER IN AN ACTION OF CONTRACT, NEGLIGENCE OR
% OTHER TORTIOUS ACTION, ARISING OUT OF OR IN CONNECTION WITH THE USE OR
% PERFORMANCE OF THIS SOFTWARE.
%==============================================================================

%- Título e Resumo ------------------------------------------------------------

\title{Atrito Cinético}
\author{
	\textsc{Amanda Canizela} \\[1ex]
	\normalsize{847501}
	\and \textsc{Cauã Diniz} \\[1ex]
	\normalsize{843988}
	\and \textsc{Júlia de Mello} \\[1ex]
	\normalsize{819881}
	\and \textsc{Lucca Pellegrini} \\[1ex]
	\normalsize{842986}
	\and \textsc{Pedro Vitor Caiafa} \\[1ex]
	\normalsize{842843}
}

\date{\today}
\renewcommand{\maketitlehookd}{
	\begin{abstract}
	\noindent
		Este experimento didático visa determinar o coeficiente de atrito
		cinético entre um bloco de madeira e uma superfície inclinada,
		utilizando uma rampa ajustada a $\SI{30}{\degree}$. Através de medições
		de tempo e posição com sensores de luz, ajustamos os dados a uma
		equação quadrática para calcular a aceleração e, subsequentemente, o
		coeficiente de atrito. Os resultados indicam um valor de $\mu_k = 0,093
		\pm 0,006$, compatível com materiais como madeira e plástico,
		destacando a eficácia do método e a importância das incertezas
		experimentais na análise física.
	\end{abstract}
}

%- Documento principal --------------------------------------------------------

\begin{document}

\maketitle

\section{Motivação}

O atrito é uma das forças mais presentes no cotidiano e desempenha papel
fundamental em diversos sistemas físicos e tecnológicos. Entender como
determinar o coeficiente de atrito cinético em um plano inclinado nos permite
modelar desde situações simples, como o movimento de objetos em superfícies
inclinadas, até aplicações em engenharia e transporte, onde a previsão do
atrito é essencial para a segurança e eficiência. O experimento realizado
possibilita, portanto, compreender de forma prática como medir esse coeficiente
e quais fatores influenciam sua determinação.

\section{Fundamentos Teóricos}

A força de atrito cinético é aquela que age sobre um corpo em movimento
relativo à superfície de apoio. Para superfícies sólidas, a experiência mostra
que essa força é praticamente constante e depende apenas das superfícies em
contato e da força normal. A força de atrito cinético é dada por:

\begin{equation*}
f_k = \mu_k N
\end{equation*}

onde $\mu_k$ é o coeficiente de atrito cinético, adimensional, determinado
experimentalmente, e $N$ é a força normal. O valor de $\mu_k$ depende das
propriedades do corpo e da superfície, sendo sempre menor que o coeficiente de
atrito estático $\mu_s$. Assim, a intensidade da força de atrito cinético é
menor que a máxima força de atrito estático que age sobre o corpo em repouso.

\section{Materiais}

Para a realização do experimento utilizamos:

\begin{itemize}
    \item Rampa com ângulo ajustável, fixada em $\SI{30 \pm 0.5}{\degree}$;
    \item Dois sensores de luz com cronômetro digital acoplado;
    \item Bloco de massa $\SI{22}{\gram}$;
	\item Régua com marcações de $\SIrange{0.05}{0.35}{\metre};$
	\item Um pequeno script em \textit{Python} (veja o Anexo~\ref{sec:python})
		para análise de dados, geração do gráfico, e ajuste de curva.
\end{itemize}

\section{Métodos}

Primeiro, medimos a massa do bloco de prova: $m = \SI{22}{\gram}$. Em seguida,
fixamos a rampa em uma inclinação de $\theta = \SI{30 \pm 0.5}{\degree}$ em
relação à horizontal, garantindo que o bloco deslize ao ser colocado na rampa,
com a superfície de madeira em contato. O bloco foi liberado a partir do
repouso sobre a rampa inclinada. Com o auxílio dos sensores de luz, medimos o
tempo $t$ que o bloco levava para percorrer intervalos de $\SI{0.05}{\metre}$,
começando do ponto inicial e repetindo para distâncias cumulativas de
$\SI{0,05}{\metre}$ até $\SI{0.35}{\metre}$. Os dados coletados foram ajustados
no script \textit{Python} a uma função linear do tipo:

\begin{align*}
	x(t) &= \frac{at^2}{2} + v_0t, \\
	x(t) &= At^2 + Bt
\end{align*}

onde consideramos $t^2$ como nossa variável independente, $A = \frac{a}{2}$, e
a posição inicial como $x(0) = \SI{0}{\meter}$.

\section{Desenvolvimento}

A tabela~\ref{tbl:dados} mostra os valores medidos de $\Delta x$, $t$ e
$t^2$.

Construímos diretamente o gráfico da posição $\Delta x$ em função do tempo ao
quadrado $t^2$, que apresenta uma forma linear, conforme esperado para movimento
com aceleração constante partindo do repouso. Do ajuste linear
(veja a figura~\ref{fig:kinetic_friction}), obtivemos:

\begin{align}\label{eq:accel}
	A &= \left( 2,056 \pm 0,066 \right) \si{\metre\per\second\squared}, \nonumber \\
	\therefore 2A &= a \Rightarrow \nonumber \\
	a &= \left( 4,1 \pm 0,1 \right) \si{\metre\per\second\squared}.
\end{align}

Sabemos que, para um corpo em um plano inclinado com atrito, o diagrama de
corpo livre mostra as forças: peso $mg$ para baixo, força normal $N$
perpendicular à superfície, força de atrito $f_k$ oposta ao movimento, e a
componente da gravidade ao longo da rampa $mg \sin\theta$. Aplicando a segunda
lei de Newton na direção da rampa: $mg \sin\theta - f_k = ma$. Como $f_k =
\mu_k N$ e $N = mg \cos\theta$, temos:

\begin{align}\label{eq:mu}
	\frac{a}{g} &= \sin\theta - \mu_k \cos\theta, \nonumber \\
	\mu_k &= \frac{\sin\theta - \frac{a}{g}}{\cos\theta}.
\end{align}

\section{Incertezas}

Para calcular a incerteza, determinamos os valores máximo e mínimo possíveis em
(\ref{eq:mu}) da seguinte forma:

\begin{align*}
	\mu_{k_\text{max}} &= \frac{\sin(\theta + \Delta\theta) - \frac{a - \Delta a}{g + \Delta g}}{\cos(\theta + \Delta\theta)}, \\
	\mu_{k_\text{min}} &= \frac{\sin(\theta - \Delta\theta) - \frac{a + \Delta a}{g - \Delta g}}{\cos(\theta - \Delta\theta)}.
\end{align*}


Substituindo os valores $g = \SI{9.8 \pm 0.05}{\meter\per\second\squared}$,
$\theta = \SI{30 \pm 0.5}{\degree}$, e os valores de $a$ e $\Delta a$ obtidos
anteriormente em (\ref{eq:accel}), temos:

\begin{align*}
	\mu_{k_\text{max}} &\approx 0,11279, \\
	\mu_{k_\text{min}} &\approx 0,07296, \\
	\overline\mu_k = \frac{\mu_{k_\text{max}} + \mu_{k_\text{min}}}{2} &\approx 0,09287, \\
	\Delta\mu_k = \frac{\mu_{k_\text{max}} - \mu_{k_\text{min}}}{2\sqrt{3}} &\approx 0,00558.
\end{align*}

\section{Resultados e Conclusão}

O valor do coeficiente de atrito cinético encontrado foi:

\begin{equation*}
	\mu_k = 0,093 \pm 0,006.
\end{equation*}

Esse valor é compatível com superfícies de madeira em contato com acrílico,
confirmando a validade do experimento. O método de ajuste linear da posição em
função do tempo ao quadrado mostrou-se eficaz para determinar a aceleração
constante, permitindo calcular $\mu_k$ com base nas equações teóricas. As
incertezas experimentais, incluindo erros na medição do ângulo e da aceleração
gravitacional, foram consideradas adequadamente. O resultado mostra que o
experimento seguiu os princípios da mecânica clássica, com boa concordância
entre teoria e prática.

%- Figuras --------------------------------------------------------------------

\onecolumn

\begin{figure}[p]
		\includegraphics[width=\columnwidth]{kinetic_friction_plot.pdf}
		\caption{
			Gráfico de posição $\Delta x$ em função do tempo ao quadrado $t^2$.
			O ajuste linear fornece $A = \SI{2,056 \pm
			0,066}{\metre\per\second\squared}$ e $B = \SI{0,037 \pm
			0,006}{\metre\per\second}$, onde $A = \frac{a}{2}$ e $B = v_0$. O
			coeficiente de determinação $R^2 = 0,9949$ indica um ajuste
			excelente, confirmando o sucesso do experimento.
		}
		\label{fig:kinetic_friction}
\end{figure}

\begin{table}[p]
	\centering
		\begin{tabular}{|| c | c | c ||}
		\hline
		$\Delta x$ ($\si{\metre}$) & $t$ ($\si{\second}$) & $t^2$ ($\si{\second^2}$) \\
		\hline\hline
		$0,05$ & $0,10225$ & $0,0104500625$ \\
		$0,10$ & $0,18040$ & $0,0325441600$ \\
		$0,15$ & $0,22880$ & $0,0522944400$ \\
		$0,20$ & $0,27200$ & $0,0739840000$ \\
		$0,25$ & $0,31830$ & $0,1012608900$ \\
		$0,30$ & $0,35640$ & $0,1270100000$ \\
		$0,35$ & $0,39620$ & $0,1569816400$ \\
		\hline
	\end{tabular}
	\caption{Valores de $\Delta x$, $t$ e $t^2$ medidos sem arredondamento.}
	\label{tbl:dados}
\end{table}

\clearpage

%- Bibliografia ---------------------------------------------------------------

%\bibliographystyle{unsrt}
%\bibliography{ref}

%- Apêndices ------------------------------------------------------------------

\appendix

\section{Código Fonte do Script Python}\label{sec:python}

Segue abaixo o script escrito para esse experimento. O módulo auxiliar
\texttt{PhysPlot}, de minha autoria, pode ser encontrado no meu
GitHub.\footnote{\url{https://github.com/lucca-pellegrini/PhysPlot/blob/v0.1.1}}

\inputminted[
	linenos = true,
	style = staroffice,
	fontsize = \footnotesize,
	frame = lines,
	framesep=2em,
	rulecolor=\color{Gray},
	label=\fbox{\color{Black}kinetic\_friction.py},
	labelposition=topline
]{python}{kinetic_friction.py}

\end{document}
